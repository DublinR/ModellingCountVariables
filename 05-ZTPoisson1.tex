\documentclass[MASTER.tex]{subfiles}
\begin{document}
%============================================================%

%============================================================%
\begin{frame}
\frametitle{Zero Truncated Poisson Distribution}
\Large

The zero-truncated Poisson (ZTP) distribution is a certain discrete probability distribution whose support is the set of positive integers. This distribution is also known as the conditional Poisson distribution[1] or the positive Poisson distribution.[2] It is the conditional probability distribution of a Poisson-distributed random variable, given that the value of the random variable is not zero. Thus it is impossible for a ZTP random variable to be zero. 

\end{frame}
%============================================================%
\begin{frame}
\frametitle{Zero Truncated Poisson Distribution}
\Large
Since the ZTP is a truncated distribution with the truncation stipulated as k > 0, one can derive the probability mass 
function g(k;λ) from a standard Poisson distribution f(k;λ) as follows: [4]
\[g(k;\lambda) = P(X = k \mid k > 0) = 
\frac{f(k;\lambda)}{1-f(0;\lambda)} = 
\frac{\lambda ^ k e^{- \lambda} }{k ! \left ( 1 - e^{- \lambda} \right )} = \frac{\lambda^k}{(e^\lambda-1)k!}\]

\end{frame}
%============================================================%
\begin{frame}
	\frametitle{Zero Truncated Poisson Distribution}
	\Large
\textbf{Distribution Parameters}\\
The mean is
\[ \operatorname{E}[X]=\frac{\lambda}{1-e^{-\lambda}}=\frac{\lambda e^\lambda}{e^\lambda-1} \]
and the variance is
\[ \operatorname{Var}[X]=\frac{\lambda}{1-e^{-\lambda}} - \frac{\lambda^2 e^{-\lambda}}{(1-e^{-\lambda})^2}\] 
\[ = \frac{\lambda e^\lambda}{e^\lambda-1}\left[1-\frac{\lambda}{e^\lambda-1}\right] \]

\end{frame}

%================================================================== %
\begin{frame}
\frametitle{Zero-Truncated Poisson regression}
\textbf{Data Set}
\begin{itemize}
	\item 
We have a hypothetical data file, \textbf{\textit{ztpoiss}} with 1,493 observations. 
\item The length of hospital stay variable is \textbf{stay}. 
\item The variable \textbf{age} gives the age group from 1 to 9 which will be treated as interval in this example. 
\item The variables \textbf{hmo} and \textbf{died} are binary indicator variables for HMO insured patients and patients who died while in the hospital, respectively.
\end{itemize}
\end{frame}

\begin{frame}[fragile]
\begin{verbatim}
##       stay            age       hmo      died   
##  Min.   : 1.00   Min.   :1.00   0:1254   0:981  
##  1st Qu.: 4.00   1st Qu.:4.00   1: 239   1:512  
##  Median : 8.00   Median :5.00                   
##  Mean   : 9.73   Mean   :5.23                   
##  3rd Qu.:13.00   3rd Qu.:6.00                   
##  Max.   :74.00   Max.   :9.00
\end{verbatim}
\end{frame}
\end{document}