R Data Analysis Examples: Zero-Truncated Negative Binomial Regression

Zero-truncated negative binomial regression is used to model count data for which the value zero cannot occur and for which over dispersion exists.

This page uses the following packages. Make sure that you can load them before trying to run the examples on this page. If you do not have a package installed, run: install.packages("packagename"), or if you see the version is out of date, run: update.packages().

require(foreign)
require(ggplot2)
require(VGAM)
require(boot)
Version info: Code for this page was tested in R Under development (unstable) (2012-11-16 r61126)
On: 2012-12-15
With: boot 1.3-7; VGAM 0.9-0; ggplot2 0.9.3; foreign 0.8-51; knitr 0.9

Please Note: The purpose of this page is to show how to use various data analysis commands. It does not cover all aspects of the research process which researchers are expected to do. In particular, it does not cover data cleaning and verification, verification of assumptions, model diagnostics and potential follow-up analyses.
