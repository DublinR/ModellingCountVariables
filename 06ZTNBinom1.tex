\documentclass[MASTER.tex]{subfiles}
\begin{document}
	

%=========================================================================%
	\begin{frame}[fragile]
			\Large
	\textbf{Zero-truncated negative binomial regression}
\large
		\begin{itemize}
\item	Zero-truncated negative binomial regression is used to model count data for which the value zero cannot occur and for which over dispersion exists.
		\end{itemize}


%This page uses the following packages. Make sure that you can load them before trying to run the examples on this page. If you do not have a package installed, run: install.packages("packagename"), or if you see the version is out of date, run: update.packages().
%
%\begin{framed}
%\begin{verbatim}
%require(foreign)
%require(ggplot2)
%require(VGAM)
%require(boot)
%\end{verbatim}
%\end{framed}
\end{frame}
%=========================================================================%
%\begin{frame}
%%
%%Version info: Code for this page was tested in R Under development (unstable) (2012-11-16 r61126)
%%On: 2012-12-15
%%With: boot 1.3-7; VGAM 0.9-0; ggplot2 0.9.3; foreign 0.8-51; knitr 0.9
%
%Please Note: The purpose of this page is to show how to use various data analysis commands. It does not cover all aspects of the research process which researchers are expected to do. In particular, it does not cover data cleaning and verification, verification of assumptions, model diagnostics and potential follow-up analyses.
%\end{frame}
%===================================================================== %
\begin{frame}
\frametitle{Zero-truncated negative binomial regression}
\begin{itemize}
\item 
	To fit the zero-truncated negative binomial model, we use the \texttt{vglm} function in the VGAM package. 
\item This function fits a very flexible class of models called vector generalized linear models to a wide range of assumed distributions. 
\item In our case, we believe the data come from the negative binomial distribution, but without zeros. 
\item Thus the values are strictly positive poisson, for which we use the positive negative binomial family via the \texttt{posnegbinomial} function passed to \texttt{vglm}.
\end{itemize}
\end{frame}
%====================================================================== %
\begin{frame}[fragile]
\frametitle{Zero-truncated negative binomial regression}
\large
\textbf{Fitting the Model with \texttt{R}}\\
We will use the \textit{hospitalstay} data again.
\begin{framed}
\begin{verbatim}
	m1 <- vglm(stay ~ age + hmo + died, 
	   family = posnegbinomial(), 
	   data = hospitalstay)

\end{verbatim}
\end{framed}
\end{frame}
%============================================================================ %
\begin{frame}[fragile]
\frametitle{Zero Truncated Negative Binomial Regression}
\begin{verbatim}
summary(m1)
## 
## Call:
## vglm(formula = stay ~ age + hmo + died, 
##     family = posnegbinomial(), 
##     data = hospitalstay)
## 
## Pearson Residuals:
##             Min    1Q Median   3Q Max
## log(munb)  -1.4 -0.70  -0.23 0.45 9.8
## log(size) -14.1 -0.27   0.45 0.76 1.0
\end{verbatim}
\end{frame}
%============================================================================ %
\begin{frame}[fragile]
	\frametitle{Zero Truncated Negative Binomial Regression}
	\begin{verbatim}
## Coefficients:
##               Estimate Std. Error z value
## (Intercept):1    2.408      0.072    33.6
## (Intercept):2    0.569      0.055    10.4
## age             -0.016      0.013    -1.2
## hmo1            -0.147      0.059    -2.5
## died1           -0.218      0.046    -4.7
\end{verbatim}
\end{frame}

%============================================================================ %
\begin{frame}
	\frametitle{Zero Truncated Negative Binomial Regression}
	\Large
	\begin{itemize}


\item The first intercept is what we know as the typical intercept. 
\item The second is the \textbf{over dispersion parameter}, $\alpha$.
\item The number of linear predictors is two, one for the expected mean $\lambda$ and one for the over dispersion.
\item Next the dispersion parameter is printed, assumed to be one after accounting for overdispersion.
\end{itemize}

\end{frame}
%============================================================================ %
\begin{frame}
	\frametitle{Zero Truncated Negative Binomial Regression}
	\Large
\begin{itemize}
\item The value of the coefficient for age, -0.0157 suggests that the log count of stay decreases by 0.0157 for each year increase in age.
\item The coefficient for hmo, -0.1471 indicates that the log count of stay for HMO patient is 0.1471 less than for non-HMO patients.
\item The log count of stay for patients who died while in the hospital was 0.2178 less than those patients who did not die.
\end{itemize}

\end{frame}
%============================================================================ %
\begin{frame}
		\frametitle{Zero Truncated Negative Binomial Regression}
	\Large
\begin{itemize}
	\item		
The value of the constant 2.4083 is the log count of the stay when all of the predictors equal zero.
\item The value of the second intercept, the over dispersion parameter, $\alpha$ is 0.5686.
\item To test whether we need to estimate over dispersion, we could fit a zero-truncated Poisson model and compare the two. (Not Covered).
\end{itemize}
\end{frame}
%============================================================================== %
\end{document}