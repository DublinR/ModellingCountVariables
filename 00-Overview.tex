\documentclass[MASTER.tex]{subfiles}
\begin{document}
\begin{frame}
	\Large
\begin{itemize}
\item This talks is about regression methods in which the dependent variable takes
nonnegative integer or count values. \item The dependent variable is usually the number of times an event occurs. 
\end{itemize}
\end{frame}
%======================================================== %
\begin{frame}
	\frametitle{Overview}
	\Large Some
examples of event counts are:
\begin{itemize}
\item number of claims per year on a particular car owner’s insurance policy,
\item number of workdays missed due to sickness of a dependent in a one-year period,
\item number of papers published per year by a researcher.
\end{itemize}
\end{frame}
%======================================================== %
\begin{frame}
\frametitle{Overview}
\Large
\begin{itemize}
\item  Poisson regression is used to model count variables.
\item  Negative Binomial regression is for modelling count variables, usually for over-dispersed count outcome variables.
\end{itemize}
\end{frame}
%================================================================== %
\end{document}






