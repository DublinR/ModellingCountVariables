\documentclass[MASTER.tex]{subfiles}
\begin{document}

%================================================================================================%
\begin{frame}[fragile]
	\frametitle{Zero-inflated Regression models}
	\Large

	\begin{itemize}
		\item Zero-inflated models attempt to account for excess zeros. 
		\item In other words, two kinds of zeros are thought to exist in the data, "true zeros" and "excess zeros". 
		\item Zero-inflated models estimate two equations simultaneously, one for the count model and one for the excess zeros.
	\end{itemize}
\end{frame}

% R Data Analysis Examples: Zero-Inflated Poisson Regression
%================================================================================================%
\begin{frame}[fragile]
	\frametitle{Zero-inflated Regression models}
	\Large
	\begin{itemize}
\item Zero-inflated poisson regression is used to model count data that has an excess of zero counts. 
\item Further, theory suggests that the excess zeros are generated by a separate process from the count values and that the excess zeros can be modeled independently. 
\item Thus, the zip model has two parts, a poisson count model and the logit model for predicting excess zeros.
\item You may want to review these Data Analysis Example pages, Poisson Regression and Logit Regression.
	\end{itemize}
\end{frame}
\end{document}