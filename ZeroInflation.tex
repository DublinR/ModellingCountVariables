\documentclass[MASTER.tex]{subfiles}
\begin{document}
	
% % FISH DATA SET
% % Zero Inflated Regression Models

%================================================================================================%
\begin{frame}[fragile]
	\frametitle{Zero-inflated Regression models}
	\Large
	\textbf{Zero-inflated Regression models - Summary}
	\begin{itemize}
		\item Zero-inflated models attempt to account for excess zeros. 
		\item In other words, two kinds of zeros are thought to exist in the data, "\textbf{true zeros}" and "\textbf{excess zeros}". 
	\end{itemize}
\end{frame}

% R Data Analysis Examples: Zero-Inflated Poisson Regression
%================================================================================================%
\begin{frame}[fragile]
	\frametitle{Zero-inflated Regression models}
	\large
	\textbf{Two Distinct Processes}
	\begin{itemize}
		\item The two parts of the a zero-inflated model are a binary model, usually a logit model to model which of the two processes the zero outcome is associated with and a count model, in this case, a negative binomial model, to model the count process. 
		%\item Zero-inflated poisson regression is used to model count data that has an excess of zero counts. 
		\item In other words, the excess zeros are generated by a separate process from the count values and that the excess zeros can be modelled independently. 
		\item Zero-inflated models estimate two equations simultaneously, one for the count model and one for the excess zeros.
\item The expected count is expressed as a combination of the two processes.
		
		% \item Thus, the zip model has two parts, a poisson count model and the logit model for predicting excess zeros.
		%\item You may want to review these Data Analysis Example pages, Poisson Regression and Logit Regression.
	\end{itemize}
\end{frame}
%=========================================================================== %
\begin{frame}
\frametitle{The \texttt{zeroinfl()} function}
\large
\begin{itemize}
\item In \texttt{R}, zero-inflated count data models can be fitted with the \texttt{zeroinfl()} function from the
\textbf{pscl} package. 
\item Both the fitting function interface and the returned model objects of class
``\texttt{zeroinfl()}`` are  modelled
after the corresponding GLM functionality in \texttt{R}. 
\end{itemize}

\end{frame}
%========================================================================== %
\begin{frame}[fragile]
\frametitle{The \texttt{zeroinfl()} function}

	The arguments of \texttt{zeroinfl(}) are given
	by
\begin{framed}
\begin{verbatim}
zeroinfl(formula, data, subset, na.action, 
       weights, offset,
       dist = "poisson", link = "logit", 
       control = zeroinfl.control(...),
       model = TRUE, y = TRUE, x = FALSE, ...)
\end{verbatim}
\end{framed}

\end{frame}

%================================================================================================%
\begin{frame}[fragile]
	\frametitle{Zero-inflated Regression models}
	\textbf{Fishing Data Set}
	%Let's pursue Example 2 from above.
	\begin{itemize}
		\item We have data on 250 groups that went to a park. 
		\item Each group was questioned about how many fish they caught (\textbf{count}), how many children were in the group (\textbf{child}), how many people were in the group (\textbf{persons}), and whether or not they brought a camper to the park (\textbf{camper}).
		\item 
		In addition to predicting the number of fish caught, there is interest in predicting the existence of excess zeros, i.e., the probability that a group caught zero fish. 
		\item We will use the variables child, persons, and camper in our model. 
	\end{itemize}
\end{frame}

%================================================================== %
\begin{frame}
\frametitle{Zero-inflated Regression models}
	\textbf{Fishing Data Set}
	%Let's pursue Example 2 from above.
	\begin{itemize}
	\item In addition to predicting the number of fish caught, there is interest in predicting the existence of excess zeros, i.e., the probability that a group caught zero fish. 
	\item We will use the variables child, persons, and camper in our model.
	\end{itemize}
\end{frame}
%================================================================== %
\begin{frame}[fragile]
	\begin{verbatim}
	> head(fish)
	nofish livebait camper persons child         xb
	1      1        0      0       1     0 -0.8963146
	2      0        1      1       1     0 -0.5583450
	3      0        1      0       1     0 -0.4017310
	4      0        1      1       2     1 -0.9562981
	5      0        1      0       1     0  0.4368910
	6      0        1      1       4     2  1.3944855
	zg count
	1  3.0504048     0
	2  1.7461489     0
	3  0.2799389     0
	4 -0.6015257     0
	5  0.5277091     1
	6 -0.7075348     0
	
	\end{verbatim}
\end{frame}
%============================================================================================= %
\begin{frame}
\frametitle{What is a Zero-Inflated Model?}
\textbf{The Fishing Example}
\begin{itemize}
\item A zero-inflated model assumes that zero outcome is due to two different processes.

\item For instance, in the example of fishing presented here, the two processes are that a subject has gone 1. \textit{ fishing} vs. 2. \textit{not fishing}.
\item If not gone fishing, the only outcome possible is zero. 
\item If gone fishing, it is then a count process. 
\end{itemize}
\begin{framed}
\[E(\mbox{no. fish caught} =k) \]\[=P(\mbox{not fishing})\times 0+P( \mbox{fishing}) \times E(y=k| \mbox{fishing}) \]
\end{framed}
\end{frame}
 
% \begin{frame}
%\frametitle{What is a Zero-Inflated Model?}
%\textbf{The Fishing Example}
%\begin{itemize}
%	\item A zero-inflated model assumes that zero outcome is due to two different processes. For instance, in the example of fishing presented here, the two processes are that a subject has gone fishing vs. not gone fishing. If not gone fishing, the only outcome possible is zero. If gone fishing, it is then a count process. The two parts of the a zero-inflated model are a binary model, usually a logit model to model which of the two processes the zero outcome is associated with and a count model, in this case, a negative binomial model, to model the count process. The expected count is expressed as a combination of the two processes. Taking the example of fishing again:
% 	
% 	\[E(n fish caught =k)=P(not gone fishing)∗0+P(gone fishing)∗E(y=k|gone fishing) \]
% \end{frame}
\end{document}