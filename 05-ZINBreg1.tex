\documentclass[MASTER.tex]{subfiles}
\begin{document}

\begin{frame}
	\LARGE
	
	\[  \mbox{PART 4 : } \]
		
	\[  \mbox{Zero-Inflated Negative Binomial Regression} \]
	

\end{frame}
\begin{frame}
\frametitle{Zero-Inflated Negative Binomial regression}	
\large
\begin{itemize} 
\item Zero-inflated negative binomial regression is for modeling count variables with excessive zeros and it is usually for overdispersed count outcome variables. 
\item Furthermore, theory suggests that the excess zeros are generated by a separate process from the count values and that the excess zeros can be modeled independently.
\end{itemize}
\end{frame}
%============================================================================ %
	\begin{frame}
	\Large
		\frametitle{Zero-Inflated Negative Binomial regression}
\begin{itemize}
\item  We are going to use the variables: \textbf{child} and \textbf{camper} to model the count in the part of negative binomial model and the variable \textbf{persons} in the logit part of the model. 
\item We use the \textbf{pscl} to run a zero-inflated negative binomial regression. 
\item We begin by estimating the model (called \texttt{m1}) with the variables of interest.
\end{itemize}	
		

\end{frame}
%============================================================================== %
\begin{frame}[fragile]
\frametitle{Zero-Inflated Negative Binomial regression}	
\large
\begin{verbatim}
		
		m1 <- zeroinfl(count ~ child + camper | persons,
			  data = fishing, dist = "negbin", 
			  EM = TRUE)
		
		summary(m1)
\end{verbatim}
\end{frame}
%============================================================================== %
\begin{frame}[fragile]
	\begin{verbatim}
		## Call:
		## zeroinfl(formula = count ~ child + camper | persons, 
		##     data = fishing, 
		##     dist = "negbin", EM = TRUE)
		## 
		## Pearson residuals:
		##    Min     1Q Median     3Q    Max 
		## -0.586 -0.462 -0.389 -0.197 18.013 
\end{verbatim}
\end{frame}
%============================================================================== %
\begin{frame}[fragile]
\frametitle{Zero-Inflated Negative Binomial regression}	
\large
	\begin{itemize}	
		\item Below the model call, you will find a block of output containing negative binomial regression coefficients for each of the variables along with standard errors, z-scores, and p-values for the coefficients. 
		\item A second block follows that corresponds to the inflation model. 
		\item This includes logit coefficients for predicting excess zeros along with their standard errors, z-scores, and p-values.
	\end{itemize}
\end{frame}
%============================================================================== %
\begin{frame}[fragile]
	\begin{verbatim}
		## Count model coefficients (negbin with log link):
		##             Estimate Std. Error z value Pr(>|z|)    
		## (Intercept)    1.371      0.256    5.35  8.6e-08 ***
		## child         -1.515      0.196   -7.75  9.4e-15 ***
		## camper1        0.879      0.269    3.26   0.0011 ** 
		## Log(theta)    -0.985      0.176   -5.60  2.1e-08 ***
\end{verbatim}
\end{frame}
%============================================================================== %
\begin{frame}[fragile]
	\begin{verbatim}
		## Zero-inflation model coefficients (binomial with logit link):
		##             Estimate Std. Error z value Pr(>|z|)  
		## (Intercept)    1.603      0.836    1.92    0.055 .
		## persons       -1.666      0.679   -2.45    0.014 *
		## ---
		## Signif. codes:  0 '***' 0.001 '**' 0.01 '*' 0.05 '.' 0.1 ' ' 1 
		## 
		## Theta = 0.373 
		## Number of iterations in BFGS optimization: 2 
		## Log-likelihood: -433 on 6 Df
\end{verbatim}
\end{frame}
\begin{frame}
	\begin{itemize}
		\item The predictors child and camper in the part of the negative binomial regression model predicting number of fish caught (count) are both significant predictors.
		\item The predictor person in the part of the logit model predicting excessive zeros is statistically significant.
		\item For these data, the expected change in log(count) for a one-unit increase in child is -1.515255 holding other variables constant. 
		\item A camper (camper = 1) has an expected log(count) of 0.879051 higher than that of a non-camper (camper = 0) holding other variables constant.
	\end{itemize}
\end{frame}

%============================================================================== %
\begin{frame}[fragile]
\frametitle{Tests of Significance}
\begin{itemize}
\item	All of the predictors in both the count and inflation portions of the model are statistically significant. 
%\item This model will fit the data significantly better than the null model, i.e., the intercept-only model. 
%\item To show that this is the case, we could compare with the current model to a null model without predictors using chi-squared test on the difference of log likelihoods. 
\end{itemize}	

\end{frame}
%%============================================================================== %
%\begin{frame}[fragile]
%	\begin{verbatim}	
%	
%	m0 <- update(m1, . ~ 1)
%	
%	pchisq(2 * (logLik(m1) - logLik(m0)), df = 3, lower.tail=FALSE)
%	
%	## 'log Lik.' 1.28e-13 (df=6)
%\end{verbatim}
%\end{frame}
%============================================================================== %
%=================================================================================================================== %
\begin{frame}
	\begin{figure}
		\centering
		\includegraphics[width=0.7\linewidth]{zinbreg10}
		
	\end{figure}
	
	%newdata1 <- expand.grid(0:3, factor(0:1), 1:4)
	%colnames(newdata1) <- c("child", "camper", "persons")
	%newdata1$phat <- predict(m1, newdata1)
	%
	%ggplot(newdata1, aes(x = child, y = phat, colour = factor(persons))) +
	%  geom_point() +
	%  geom_line() +
	%  facet_wrap(~camper) +
	%  labs(x = "Number of Children", y = "Predicted Fish Caught")
	
\end{frame}
\begin{frame}
	\begin{itemize}
		\item The log odds of being an excessive zero would decrease by 1.67 for every additional person in the group. 
		\item In other words, the more people in the group the less likely that the zero would be due to not gone fishing. 
		\item Put plainly, the larger the group the person was in, the more likely that the person went fishing.
		\item The Vuong test suggests that the zero-inflated negative binomial model is a significant improvement over a standard negative binomial model. 
	\end{itemize}
\end{frame}
\end{document}	