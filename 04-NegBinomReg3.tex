\documentclass[MASTER.tex]{subfiles}
\begin{document}
%================================================ %
\begin{frame}[fragile]
	\frametitle{Negative Binomial Regression with \texttt{R} }
	\Large
	
	\textbf{Things to consider}
	\begin{itemize}
	\item It is not recommended that negative binomial models be applied to small samples.
	\item One common cause of over-dispersion is excess zeros by an additional data generating process. 
	In this situation, \textbf{zero-inflated model} should be considered.
	\item	
	If the data generating process does not allow for any 0s (such as the number of days spent in the hospital), then a \textbf{zero-truncated model} may be more appropriate.
	\end{itemize}
\end{frame}

%================================================ %
\begin{frame}[fragile]
	\frametitle{Negative Binomial Regression with \texttt{R} }
	\Large
	\begin{itemize}
	\item
	Count data often have an exposure variable, which indicates the number of times the event could have happened. 
	\item This variable should be incorporated into your negative binomial regression model with the 
	use of the \texttt{offset} option. 
	%See the glm documentation for details.
	\end{itemize}
\end{frame}
%================================================ %
\begin{frame}[fragile]
	\frametitle{Negative Binomial Regression with \texttt{R} }
	\Large
		\begin{itemize}
	\item
	The outcome variable in a negative binomial regression cannot have negative numbers.
	\item You will need to use the \texttt{m1\$resid} command to obtain the residuals from our model to check 
	other assumptions of the negative binomial model 
	% (see Cameron and Trivedi (1998) and Dupont (2002) for more information).
	\end{itemize}
\end{frame}


%================================================== %	
\end{document}
